\documentclass[12pt]{article}
\usepackage{a4wide}
\usepackage[french,english]{babel}
\usepackage[utf8]{inputenc}
\usepackage[T1]{fontenc}
\usepackage{amsthm}
\usepackage{amssymb}
\usepackage{amsmath}
\usepackage{stmaryrd}
\usepackage{makeidx}
\usepackage{graphicx}
\usepackage{caption}
\theoremstyle{plain}
\usepackage[colorlinks=true,breaklinks=true,linkcolor=black]{hyperref} %pour faire des liens hypertextes
\usepackage{amsfonts}
\newtheorem{thm}{Theorem}[section] %[subsection sert à numéroter les théorèmes d'après la sous-section où ils se trouvent
\newtheorem{prop}[thm]{Proposition}
\newtheorem{cor}[thm]{Corollary}
\newtheorem{defin}[thm]{Définition}
\newtheorem{lemme}[thm]{Lemma}
\date{}


\makeindex

\usepackage{mathrsfs}
\usepackage[all]{xy}
\usepackage{mathbbol}


\theoremstyle{definition}
\newtheorem{defn}[thm]{Définition}
\newtheorem{defn/prop}[thm]{Définition/Proposition}
\newtheorem{defn/thm}[thm]{Définition/Théorème}
\newtheorem{conj}[thm]{Conjecture}
\newtheorem{exmp}[thm]{Exemple}
\newtheorem{rque}[thm]{Remark}
\usepackage{geometry}
\geometry{hmargin=2.5cm,vmargin=1.5cm}
\title{Gindikin-Karpelevich finiteness}
\bibliographystyle{alpha}%pour que les références soient une partie du nom+l'année
\author{Auguste \textsc{Hébert}}
\makeatletter \@addtoreset{figure}{section}\makeatother
\renewcommand\thefigure{\thesection.\arabic{figure}}



\newcommand{\R}{\mathbb{R}}
\newcommand{\A}{\mathbb{A}}
\newcommand{\N}{\mathbb{Z}_{\geq 0}}
\newcommand{\Z}{\mathbb{Z}}
\newcommand{\Q}{\mathbb{Q}}
\newcommand{\C}{\mathbb{C}}
\newcommand{\Ne}{\mathbb{Z}_{\geq 1}}
\newcommand{\GL}{\mathrm{GL}}
\newcommand{\I}{\mathcal{I}}
\newcommand{\T}{\mathcal{T}}
\newcommand{\Id}{\mathrm{Id}}


\begin{document}
\maketitle


\section{Introduction}

Let $\mathcal{G}$ a split Kac-Moody group over an ultrametric field. By \cite{gaussent2008kac} and \cite{rousseau2012almost} one can associate a hovel $\I$ to $\mathcal{G}$, which has some properties. This is some kind of generalization of Bruhat-Tits buildings for reductive groups. Studying $\I$ enables to get information on $\mathcal{G}$. Let $(C,X,Y,(\alpha_i)_{i\in I},(\alpha_i^\vee)_{i\in I})$ be a generating root system associated to $\mathcal{G}$.  Let $\A=Y\otimes \R$. We will call $\A$ the standard apartment of $\I$. The hovel is a set covered with vectorial spaces isomorphic to $\A$ analogously to euclidean buildings.

Let $C_f^v=\{x\in \A|\alpha_i(x)>0\forall i\in I\}$. As in the case of euclidean buildings we can define a retraction $\rho_{+\infty}$ (resp. $\rho_{-\infty}$) of $\I$ on $\A$ with center $C_f^v$ (resp. $-C_f^v$). Let $Q^\vee=\bigoplus_{i\in I} \Z \alpha_i^\vee$ and $\A_{in}$ the inessential part of $\A$ : $\A_{in}=\{x\in \A |\alpha_i(x)=0\ \forall i\in I\}$. We can define a vectorial distance $d^v$ on a subset of $\I^2$ with values in $\overline{C_f^v}$. For $\lambda\in \overline{C_f^v}$, $B^v(0,\lambda)=\{x\in\I|d^v(0,x)\mathrm{\ is\ defined\ and\ }d^v(0,x)=\lambda\}$.

\vspace{3mm}
 The aim of this article is to show the following four theorems : 
\vspace{3mm}

Theorem~\ref{thm inclusion} :
Let $\mu\in \A$. Then if $\mu\notin Q_{\R}^\vee$, $\rho_{+\infty}^{-1}(\{\lambda+\mu\})\cap \rho_{-\infty}^{-1}(\{\lambda\})$ is empty for all $\lambda\in \A$. If $\mu\in Q^\vee_{\R}$, then for $\lambda\in \A$ sufficiently dominant, $\rho_{+\infty}^{-1}(\{\lambda+\mu\})\cap \rho_{-\infty}^{-1}(\{\lambda\})\subset B^v(0,\lambda)\cap \rho_{+\infty}^{-1}(\{\lambda+\mu\}) $. 

\vspace{3mm}
Corollary~\ref{cor finitude des boules} :
 Let $\mu\in Y+\A_{in}$. For all $\lambda\in Y+\A_{in}$ , $B^v(0,\lambda)\cap \rho_{+\infty}^{-1}(\{\mu+\lambda\})$ is finite and is empty if $\mu\notin Q^\vee_-=\bigoplus _{i\in I}\Z_{\leq 0}\alpha_i^\vee$.

\vspace{3mm}
Theorem~\ref{thm invariance des cardinaux} : 
Let $\mu\in \A$ and $\lambda\in Y+\A_{in}$. Then $|\rho_{+\infty}^{-1}(\{\lambda+\mu\})\cap\rho_{-\infty}^{-1}(\{\lambda\})|=|\rho_{+\infty}^{-1}(\{\mu\})\cap\rho_{-\infty}^{-1}(\{0\})|$. Therefore theses sets are finite and if $\mu\notin Q^\vee_-$ they are empty (this is formulated slightly differently in the article).

\vspace{3mm}
Theorem~\ref{thm égalité des ensembles bis} :
 Let $\mu\in Q^\vee$. Then for $\lambda\in Y^{++}+\A_{in}$ sufficiently dominant, $B^v(0,\lambda)\cap \rho_{+\infty}^{-1}(\{\lambda+\mu\})=\rho_{-\infty}^{-1}(\{\lambda\})\cap\rho_{+\infty}^{-1}(\{\lambda+\mu\})$.
\vspace{3mm}

These theorems generalize theorem 1.8 of \cite{braverman2014affine} because we do not require that $\mathcal{G}$ is affine. Corollary~\ref{cor finitude des boules} is a slight generalization of results from part 5 of\cite{gaussent2008kac}.
\vspace{3mm}


The main tools used to show them are Hecke paths, which are defined and used in \cite{gaussent2008kac} part 5 and \cite{gaussent2014spherical} part 1.8 and which correspond to the image of segments of $\mathcal{I}$ by retractions centred at $\pm\infty$, and results of \cite{gaussent2014spherical}.

In section~\ref{sect general frameworks} we set the general frameworks and the notation and we define hovels.

 In section~\ref{sect preliminaries} we first define two applications: $T_\nu:\I\rightarrow\R_+$ and $y_\nu:\I\rightarrow \A$, for fixed $\nu\in C_f^v$. For $x\in\I$, $T_\nu(x)$ measures the distance between a point and $\A$ along $\R_+\nu$ and $y_\nu(x)$ defines the projection of $x$ on $\A$ along $\R_+\nu$. We also determine the antecedents of some kinds paths by $\rho_{-\infty}$. 
 
 In section~\ref{sect bounding of T}, we show that under some conditions, $T_\nu$ is bounded and we deduce theorem~\ref{thm inclusion}. 
 
 In section~\ref{sect translations}, we study some kinds of translation of $\I$, which enables to show corollary~\ref{cor finitude des boules} and theorem~\ref{thm invariance des cardinaux}.
 
 In section~\ref{sect Y^++}, we show that $Y^{++}=Y\cap \overline{C_f^v}$ is a finitely generated monoid.
 
 In section~\ref{sect proof of final theorem}, we use the tools of preceding sections to show theorem~\ref{thm égalité des ensembles bis}. 


\section{General frameworks}\label{sect general frameworks}
\subsection{Root generating system}
A Kac-Moody matrix (or generalized Cartan matrix) is a square matrix $C=(c_{i,j})_{i,j\in I}$ with integers coefficients, indexed by  a finite set $I$ and satisfying : 
\begin{enumerate}
\item $\forall i\in I,\ c_{i,i}=2$

\item $\forall (i,j)\in I^2|i \neq j,\ c_{i,j}\leq 0$

\item $\forall (i,j)\in I^2,\ c_{i,j}=0 \Leftrightarrow c_{j,i}=0$.
\end{enumerate}

A root generating system is a $5$-tuple $\mathcal{S}=(C,X,Y,(\alpha_i)_{i\in I},(\alpha_i^\vee)_{i\in I})$ made of a Kac-Moody matrix $C$ indexed by $I$, of two dual free $\Z$-modules $X$ (of characters) and $Y$ (of cocharacters) of finite rank $\mathrm{rk}(X)$, a family $(\alpha_i)_{i\in I}$ (of simple roots) in $X$ and a family $(\alpha_i^\vee)_{i\in I}$ (of simple coroots) in $Y$. They have to satisfy the following compatibility condition : $c_{i,j}=\alpha_j(\alpha_i^\vee)$ for all $i,j\in I$. We also suppose that the family $(\alpha_i)_{i\in I}$ is free in $X$ and that the family $(\alpha_i^\vee)_{i\in I}$ is free in $Y$.



 We know fix a Kac-Moody matrix $C$ and a root generating system with matrix $C$.



Let $V=Y\otimes \R$. Every element of $X$ induces a linear form on $V$. We will consider $X$ as a subset of the dual $V^*$ of $V$ : the $\alpha_i$, $i\in I$ are viewed à vérifier as linear form on $V$. For $i\in I$, we define an involution $r_i$ of $V$ by $r_i(v)=v-\alpha_i(v)\alpha_i^\vee$ for all $v\in V$. This is a reflection of hyperplane $\ker \alpha_i$. The subgroup of $\mathrm{GL}(V)$ generated by the $\alpha_i$ for $i\in I$ is denoted by $W^v$ and is called the Weyl group of $\mathcal S$.


For $x\in V$ we let $\alpha(x)=(\alpha_i)_{i\in I}\in \R^I$.

Let  $Q^\vee=\bigoplus_{i\in I}\Z\alpha_i^\vee$ and $P^\vee=\{y\in V |\alpha(v)\in \Z^I\}$. We call  $Q^\vee$ the \textit{coroot-lattice} and $P^\vee$ the \textit{root-lattice}. Let $Q_+^\vee=\bigoplus_{i\in I}\N \alpha_i^\vee$ and $Q^\vee_{\mathbb{R}}=\bigoplus_{i\in I}\R\alpha_i^\vee$. This enables us to define a preorder $\leq_{Q^\vee}$ on $V$ by the following way : for all $x,y\in V$, one writes $x\leq_{Q^\vee}y$ if $y-x\in Q^\vee_+$. 




Let $V_{in}=\bigcap_{i\in I}\ker \alpha_i$. Then one has $Y+V_{in}\subset P^\vee$. 


\begin{rque}
Suppose $Y+V_{in}\varsubsetneq P^\vee$. Then one can construct a generating root system $(C,X,Y,(\alpha_i')_{i\in I},(\alpha_i'^\vee)_{i\in I})$ such that $P'^\vee=\{x\in V|\forall i\in I, \alpha_i(x)\in \Z\}=Y+V_{in}$ by the following way.
 Let $n=\mathrm{rk} X$ and suppose $I\subset\llbracket 1,n\rrbracket$. Let $e_1,\ldots,e_n$ a $\mathbb{Z}$-basis of $X$. For $i\in I$ one writes 
$\alpha_i^\vee=\sum\beta_{i,j}e_j$. For $i\in I$, with $\beta_{i,j}\in \Z$. Let 
$d_i=\gcd\big((\beta_{i,j})_{j\in \llbracket 1 ,n\rrbracket}\big)$,  $\alpha_i'=\frac{\alpha_i}{d_i}$ and $\alpha_i^\vee=d_i\alpha_i^\vee$. Then one can complete $(\alpha_i')_{i\in I}$ in a $\Z$-basis $(\alpha_i')_{i\in \llbracket 1, n\rrbracket} $ of $X$. Let $x\in \Z^n$. There exists $u\in Y$ such that $(\alpha'_i(u))_{i\in \llbracket 1,n\rrbracket}=x$ and in particular, for all $x\in \Z^I$, there exists $u\in Y$ such that $(\alpha'_i(u))_{i\in I}=x$, and therefore $P'^\vee=\{x\in V|\forall i\in I, \alpha_i(x)\in \Z\}=Y+V_{in}.$
\end{rque}

\subsection{Vectorial faces}

Define $C_f^v=\{u\in V|\ \alpha_i(v)>0\}$. We call it the \textit{fundamental chamber}. For $J\subset I$, one sets $F^v(J)=\{v\in V|\alpha_i(v)=0\ \forall i\in J,\alpha_i(v)>0\ \forall i\in J\backslash I\}$. Then the closure $\overline{C_f^v}$ of $C_f^v$ is the union of the $F^v(J)$ for $J\subset I$. The \textit{positive} (resp. \textit{negative}) \textit{vectorial faces} are the sets $w.F^v(J)$ for $w\in W^v$ and $J\subset I$.  We call \textit{positive chamber} (resp. \textit{negative}) every cone  of the shape $w.C_f^v$ for some $w\in W^v$.  For all $x\in C_f^v$ and for all $w\in W^v$, $w.x=x$ implies that $w=1$. In particular the action of $w$ on the positive chambers is simply transitive. The \textit{Tits cone} $\mathcal T$ is defined by $\mathcal{T}=\bigcup_{w\in W^v} w.\overline{C^v_f}$. We also consider the negative cone $-\mathcal{T}$.
We define a $W^v$ invariant relation $\leq$ on $V$ by : $\forall (x,y)\in V\mathrm{}^2$, $x\leq y\ \Leftrightarrow\ y-x\in \mathcal{T}$.


\subsection{Filters}

\begin{defin}
A filter in a set $E$ is a nonempty set $F$ of nonempty subsets of $E$ such that, if $S$, $S'\in F$ then $S\cap S'\in F$ and, if $S'\subset S$, with $S\in F$ then $S'\in F$.
\end{defin}

If $F$ is a filter in a set $E$, and $E'$ is a subset of $E$, one says that $F$ contains $E'$ if every element of $F$ contains $E'$. If $E'$ is nonempty, the set $F_{E'}$ of subsets of $E$ containing $E'$ is a filter. By abuse of language, we will sometimes say that $E'$ is a filter by identifying $F_{E'}$ and $E'$. If $F$ is a filter in $E$, its closure $\overline F$ (resp. its convex envelope) is the filter of subsets of $E$ containing the closure (resp. the convex envelope) of some element of $F$.

If $x\in V\mathrm{}$ and $\Omega$ is a subset of $V$ containing $x$ in its closure, then the \textit{germ} of $\Omega$ in $x$ is the filter $germ_x(\Omega)$ of subsets of $V$ containing a neighbourhood of $x\in \Omega$.

A \textit{sector} in $V$ a set of the shape $\mathfrak{s}=x+C^v$ with $C_f^v=\pm w.C_f^v$ for some $x\in \mathbb{A}$ and $w\in W^v$. The point $x$ is its \textit{base point} and $C^v$ is its \textit{direction}. The intersection of two sectors of the same direction contains a sector of the same direction.



A \textit{sector-germ} of a sector $\mathfrak{s}=x+C^v$ is the filter $\mathfrak{S}$ of subsets of $V$ containing a $V$-translate of $\mathfrak{s}$. It only depends on the direction $C^v$. We denote by $+\infty$ (resp. $-\infty$) the sector-germ of $C_f^v$ (resp. of $-C_f^v$).


One defines an action of the group $W^v$ on $V^*$ by the following way : if $x\in V$, $w\in W^v$ and $\alpha\in V^*$ then $(w.\alpha)(x)=\alpha(w^{-1}.x)$. Let $\phi=\{w.\alpha_i|(w,i)\in W^v\times I\}$. Then $\Phi\subset \sum_{i\in I}\Z\alpha_i$. For $\alpha\in \Phi$ and $k\in \mathbb{Z}$, let $M(\alpha,k)=\{x\in V\mathrm{}|\alpha(x)=k\}$. Let $W^a$ the subgroup of $\mathrm{GA}(V\mathrm{})$ of elements stabilizing $\bigcup_{\alpha \in \Phi, k\in \Z} M(\alpha,k)$, where $\mathrm{GA}(V\mathrm{})$ is the set of affine isomorphisms of $V$. Then $W^a\supset W^v$.


A ray $\delta$ with base point $x$ and containing $y\neq x$ (or the interval $]x,y]=[x,y]\backslash\{x\}$ or $[x,y]$) is called \textit{preordered} if $x\leq y$ or $y\leq x$ and \textit{generic} if $y-x\in \pm\mathring \T$. 



In the next subsection, we define the notions of faces, enclosures and of chimneys defined in \cite{rousseau2011masures} 1.7 and 1.10 and in  \cite{gaussent2014spherical} 1.4. For a first reading, one can just know the following facts about these objects and skip this subsection :

\begin{enumerate}

\item To any filter $F$ of $V$ is associated its enclosure $\mathrm{cl}_{\mathbb{A}}(F)$ which is a filter in $\mathbb{A}$ containing the convex envelope of the closure of $F$.

\item A face or a chimney is a filter in $V$.



\item A sector is a chimney which is solid and splayed.

\item The germ of a sector viewed as a chimney is the germ of the sector.

\item Every $x\in V\mathrm{}$ is in some face of $V$.




\item The group $W^a$ permutes the sectors, the enclosures, the faces and the chimneys of $V$.\label{fait sur les faces}

\end{enumerate}


\subsection{Definitions of enclosures, faces, chimneys and related notions}


Let $\Delta=\phi\cup\Delta_{im}^+\cup\Delta_{im}^-$ be the set of all roots, defined in  \cite{kac1994infinite}. One has $\Delta\subset V^*$. For $\alpha\in \Delta$, and $k\in \Z\cup{+\infty}$, let $D(\alpha,k)=\{v\in V| \alpha(v)+k\geq 0\}$ (and $D(\alpha,+\infty)=V\mathrm{}$ for all $\alpha\in \Delta$) and $D^\circ(\alpha,k)=\{v\in V| \alpha(v)+k > 0\}$ (for $\alpha\in \Delta$ and $k\in \Z\cup\{+\infty\}$).

Given a filter $F$ of subsets of $V$, its \textit{enclosure} $\mathrm{cl}_V\mathrm{}(F)$ is the filter made of subsets of $V$ containing an element of $F$ of the shape $\bigcap_{\alpha\in \Delta}D(\alpha,k_\alpha)$ where $k_\alpha\in \Z\cup\{+\infty\}$ for all $\alpha\in \Delta$.

A \textit{face} $F$ in an apartment $V$ is a filter associated to a point $x\in V\mathrm{}$ and a vectorial face $F^v\in V$. More precisely, a subset $S$ of $V$ is an element of the face $F=F(x,F^v)$ if and only if, it contains an intersection of half-spaces $D(\alpha,k_\alpha)$ or open half-spaces $D^\circ(\alpha,k_\alpha)$, with $k_\alpha\in \Z$ for all $\alpha\in \Delta$, that contains $\Omega\cap (x+F^v)$, where $\Omega$ is an open neighbourhood of $x$ in $V$.

There is an order on the faces: if $F\subset \overline{F'}$ we say that "$F$ is a face of $F'$" or "$F'$ contains $F$". The dimension of a face $F$ is the smallest dimension of an affine space generated by some $S\in F$. Such an affine space is unique and is called its support. A face is said to be \textit{spherical} if the direction of its support meets the open Tits cone $\mathring \T$; then its pointwise stabilizer $W_F$ in $W$ is finite.

A \textit{chamber} (or alcove) is a maximal face, or equivalently, a face such that all its elements contains a nonempty open subset of $V$.

A \textit{panel} is a spherical face maximal among faces that are not chambers or, equivalently, a spherical face of dimension $n-1$.

A \textit{chimney} in $V$ is associated to a face $F=F(x,F_0^v)$ and to a vectorial face $F^v$; it is the filter $\tau(F,F^v)=\mathrm{cl}_\mathbb{A}(F+F^v)$. The face $F$ is the basis of the chimney and the vectorial face $F^v$ its direction. A chimney is \textit{splayed} if $F^v$ is spherical, and it is \textit{solid} if its support (as a filter, i.e., the smallest affine subspace  containing $\tau$) has a finite pointwise stabilizer in $W^v$. As splayed chimney is therefore solid. 

A \textit{shortening} of a chimney $\tau(F,F^v)$, with $F=F(x,F_0^v)$ is a chimney of the shape $\tau \big((F(x+\xi),F_0^v),F^v\big)$ with $\xi\in \overline{F^v}$ (this definition is slightly different from the one of \cite{rousseau2011masures} 1.12 but follows \cite{rousseau2012almost} 3.6). The \textit{germ} of a chimney $\tau$ is the filter of subsets of $V$ containing a shortening of $\tau$.




\subsection{Hovel}
We now denote by $\A$ the affine space $V$ equipped with its faces, chimneys, ...


An apartment of type $\A$ is a set $A$ with a nonempty set $\mathrm{Isom}(\A,A)$ of bijections (called isomorphisms) such that if $f_0\in \mathrm{Isom}(\A,A)$ then $f\in \mathrm{Isom}(\A,A)$ if and only if, there exists $W\in W^a$ satisfying $f=f_0\circ w$. An isomorphism between two apartments $\phi:A\rightarrow A'$ is a bijection such that ($f\in \mathrm{Isom}(\mathbb{A},A)$ if, and only if, $\phi \circ f\in \mathrm{Isom}(\A,A')$). We extend all the notions that are preserved by $W^a$ to each apartment. By the fact~\ref{fait sur les faces} of the above subsection, sectors, enclosures, faces and chimney are well defined in any apartment of type $\A$.

\begin{defn}
An ordered affine hovel of type $\A$ is a set $\mathcal{I}$ endowed with a covering $\mathcal{A}$ of subsets called apartments such that : 

(MA1) Any $A\in \mathcal{A}$ admits a structure of an apartment of type $\A$.

(MA2) If $F$ is a point, a germ of a preordered interval, a generic ray or a solid chimney in an apartment $A$ and if $A'$ is another apartment containing $F$, then $A\cap A'$ contains the enclosure $\mathrm{cl}_A(F)$ of $F$ and there exists an isomorphism from $A$ onto $A'$ fixing $\mathrm{cl}_A(F)$.

(MA3) If $\mathfrak{R}$ is the germ of a splayed chimney and if $F$ is a face or a germ of a solid chimney, then there exists an apartment that contains $\mathfrak{R}$ and $F$.

(MA4) If two apartments $A$, $A'$ contain $\mathfrak{R}$ and $F$ as in (MA3), then there exists an isomorphism from $A$ to $A'$ fixing $\mathrm{cl}_A(\mathfrak{R}\cup F)$.

(MAO) If $x$, $y$ are two points contained in two apartments $A$ and $A'$, and if $x\leq_{A} y$ then the two segments $[x,y]_A$ and $[x,y]_{A'}$ are equal.
\end{defn}


In this definition, we say that an apartment contains a germ of a filter if it contains at least one element of this germ. We say that an application fixes a germ if it fixes at least one element of this germ.

\begin{rque}(consequence 2.2.3) of \cite{rousseau2011masures})\label{rque axioms MA3 et MA4 modifiés}
By (MA2), the axioms (MA3) and (MA4) also apply in a hovel when $F$ is a point, a germ of a preodered segment and when $\mathfrak{R}$ or $F$ is a germ of a generic ray or a germ of a spherical sector face.
\end{rque}

\vspace{3mm}
Until the end of this article, $\I$ will be an affine ordered hovel. We suppose that $\I$ is thick of \textit{finite thickness}: the number of chambers (=alcoves) containing a given panel has to be finite, greater or equal to $3$. This assumption will be crucial to use some theorems of \cite{gaussent2014spherical} but we will not use it directly. 

We assume that $\I$ has a strongly transitive group of automorphisms $G$, which means that all isomorphisms involved in the above axioms are induced by elements of $G$. We choose in $\I$ a fundamental apartment, that we identify with $\A$. As $G$ is strongly transitive, the apartments of $\I$ are the sets $g.\A$ for $g\in G$. The stabilizer $N$ of $\A$ induces a group $\nu(N)$ of affine automorphisms of $\A$ and we suppose that $\nu(N)=W^v\ltimes Y$.

An example of such an hovel $\I$ is the hovel associated to a split Kac-Moody group over an ultrametric field constructed in $\cite{gaussent2008kac}$ and in \cite{rousseau2012almost}.

\paragraph{Vectorial distance}
For $x\in \mathcal{T}$, we denote by $x^{++}$ the unique element in $\overline{C^v_f}$ conjugated by $W^v$ to $x$. Let $\I\times_{\leq}\I=\{(x,y)\in \I^2|x\leq y\}$ be the set of increasing pairs in $\I$. Such a pair is always in the same apartment $g.\A$ for some $g\in G$. So $g^{-1}.y-g^{-1}.x\in \mathcal{T}$, and we define the \textit{vectorial distance} $d^v(x,y)\in\overline{C_f^v}$ by $d^v(x,y)=(g^{-1}.y-g^{-1}.x)^{++}$. It does not depend on the choices we made.

For $x\in \I$ and $\lambda\in \overline{C_f^v}$, one defines $B^v(x,\lambda)=\{y\in \I|x\leq y\mathrm{\ and\ }d^v(x,y)=\lambda\}$.

\begin{rque}\label{rque caractérisation distance vectorielle}
a) If $a\in Y$ and $\lambda\in \overline{C_f^v}$, then $B^v(a,\lambda)=\{x\in \I|\exists g\in G|g.a=a\mathrm{\ and\ }g.x=a+\lambda\}$.

b) Let $x,y\in \I$ and suppose that for some $g\in G$, $g.y-g.x\in \overline{C_f^v}$. Then $x\leq y$ and $d^v(x,y)=g.y-g.x$.
\end{rque}

\subsection{Retractions and Hecke paths}


Let  $\mathfrak{R}$ be the germ of a splayed chimney of an apartment $A$. Let $x\in \I$. By (MA3), for all $x\in \I$, there exists an apartment $A_x$ of $\I$ containing $x$ and $\mathfrak{R}$. By (MA4), there exists an apartment $\phi:A_x\rightarrow A$ fixing $\mathfrak{R}$. By \cite{rousseau2011masures} 2.6, $\phi(x)$ does not depend of the choices we made and thus we can let $\rho_{A,\mathfrak{R}}(x)=\phi(x)$.

The application $\rho_{A,\mathfrak{R}}$ is a retraction from $\I$ on $A$. It only depends on $\mathfrak{R}$ and $A$ and we call it the \textit{retraction from $\I$ on $A$ of center $\mathfrak{R}$}. 

We denote by $\rho_{+\infty}$ (resp. $\rho_{-\infty}$ ) the retraction on $\A$ of center $+\infty$ (resp. $-\infty$).

We now defines Hecke paths. They are more or less the images by $\rho_{-\infty}$ of segments $[x,y]$ in $\I$. The definition is a bit technical but it expresses the fact that the image of such a path "goes nearer $+\infty$" when it crosses a wall. A consequence of that is remark~\ref{rque chemins de Hecke} and we will not use directly this definition in the following.

 We consider piecewise linear continuous paths $\pi:[0,1]\rightarrow \A$ such that the values  of $\pi'$ belong to some orbit $W^v.\lambda$ for some $\lambda\in \overline{C_f^v}$. Such a path is called a $\lambda$-\textit{path}. It is increasing with respect to the preorder relation $\leq$ on $\A$. For any $t\neq 0$ (resp. $t\neq 1$), we let $\pi'_-(t)$ (resp. $\pi'_+(t))$ denote the derivative of $\pi$ at $t$ from the left (resp. from the right).

\begin{defn}
A Hecke path of shape $\lambda$ with respect to $-C_f^v$ is a $\lambda$-path such that
 $\pi'_+(t)\leq_{W^v_{\pi (t)}} \pi'_-(t)$ for all $t\in [0,1]\backslash \{0,1\}$, which 
 means that there exists a $W_{\pi(t)}^v$-chain from $\pi'_-(t)$ to $\pi'_{+}(t)$, i.e., a 
 finite sequence $(\xi_0=\pi'_-(t),\xi_1,\ldots, \xi_s=\pi'_+(t))$ of vectors in $V$ and
  $(\beta_1,\ldots,\beta_s)\in \phi^s$ such that, for all $i\in \llbracket 1,s\rrbracket$,
\begin{enumerate}
\item $r_{\beta_i}(\xi_{i-1})=\xi_i.$

\item $\beta_i(\xi_{i-1})<0.$

\item $r_{\beta_i}\in W^v_{\pi(t)}$; i.e., $\beta_i(\pi(t))\in \Z$: $\pi(t)$ is in a wall of direction $\ker(\beta_i)$.

\item Each $\beta_i$ is positive with respect to $-C_f^v$; i.e., $\beta_i(C_f^v)>0$.
\end{enumerate}
\end{defn}

\begin{rque}\label{rque chemins de Hecke}
Let $\pi:[0,1]\rightarrow \A$ a Hecke path of shape $\lambda\in \overline{C_f^v}$ with respect to $+\infty$. Then if $t\in [0,1]$ such that $\pi$ is derivable in $t$ and $\pi'(t)\in \overline{C_f^v}$, then for all $s\geq t$, $\pi$ is derivable in $s$ and $\pi'(s)=\lambda$.
\end{rque}

\section{Preliminaries}\label{sect preliminaries}

In this part we begin by defining for all $\nu\in C_f^v$ two applications $y_\nu:\I\rightarrow \A$ and $T_\nu:\I\rightarrow \R_+$, where for all $x\in \I$, $T_\nu(x)$ and $y_\nu(x)$ can be considered as the distance between $x$ and $\A$ along $\R_+\nu$ and the projection of $x$ on $\A$ along $\R_+\nu$. 

We also show that the only antecedent of some paths for $\rho_{-\infty}$ are themselves (this is lemma~\ref{lemme image réciproque de segments}).

\vspace{3mm}

Let us sketch the proof of theorem~\ref{thm inclusion}. We fix $\nu\in C_f^v$ and $\mu\in Q^\vee_\R$. Let $T^-=T_{\nu}^-$ and $y^-=y_\nu^-$ be the analogous of $T_\nu$ and $y_\nu$ with $\nu$ replaced by $-\nu$. We show that if $\lambda\in \A$, $T^-\big(\rho_{+\infty}^{-1}(\{\lambda+\mu\})\cap \rho_{-\infty}^{-1}(\{\lambda\})\big)$ is bounded by some constant $h(\mu)$ (this is corollary~\ref{corollaire majoration de T}), which uses lemma~\ref{lemme image réciproque de segments} and lemma~\ref{lemme fin des chemins de Hecke longs}). Therefore, for $\lambda$ sufficiently dominant, $y^-\big(\rho_{+\infty}^{-1}(\{\lambda+\mu\})\cap \rho_{-\infty}^{-1}(\{\lambda\})\big)\subset C_f^v$ and lemma~\ref{lemme_rétraction et distance vectorielle} complete the proof.

\vspace{3mm}
Let us recall briefly the notion of parallelism in $\I$. This is done more completely in \cite{rousseau2011masures} part 3. Let $\delta$ and $\delta'$ be two generic rays in $\I$. Then there exists a splayed chimney $R$ containing $\delta$ and a solid chimney $F$ containing $\delta'$. By (MA3) there exists an apartment $A$ containing the germ $\mathfrak{R}$ of $R$ and $F$. Therefore $A$ contains translates of $\delta$ and $\delta'$ and we say that $\delta$ and $\delta'$ are \textit{parallel}, if these translates are parallel in $A$. Parallelism is an equivalence relation and its equivalence classes are called \textit{directions}.

\begin{lemme}\label{lemme demi-droite de base donnée}
Let $x\in \I$ and $\delta$ be a generic ray. Then there exists a unique $ray$ $x+\delta$ in $\I$ with base point $x$ and direction $\delta$. In any apartment $A$ containing $x$ and a ray $\delta'$ parallel to $\delta$, this ray is the translate in $A$ of $\delta'$ having $x$ as a base point.
\end{lemme}

This lemme is analogous to proposition 4.7 1) of \cite{rousseau2011masures}. The difficult part of this lemma is the uniqueness of such a ray because second assertion of the term of the lemma yields a way to construct a ray having direction $\delta$ and $x$ as a base point. This uniqueness can be shown exactly in the same manner as the proof of proposition 4.7.1) by replacing "spherical sector face" by "generic ray". This is possible by NB.a) of proposition 2.7 and by 2.2 3) of \cite{rousseau2011masures}.


\paragraph{Definition of $y_\nu$ and $T_\nu$ (resp. $y^-_\nu$ and $T^-_\nu$)}
Let $x\in \mathcal{I}$. Let $\nu \in C_f^v$ and $\delta=\R_+\nu$, which is a generic ray. According to axiom (MA3) applied to a face containing $x$ and the splayed chimney $C^v_f$, there exists an apartment $A$ containing $x$ and $+\infty$. Then $A$ contains $x+\delta$. Then $x+\delta\cap \A$ is nonempty. Let $z\in x+\delta\cap \A$. Then $A\cap \A$ contains $z$, $+\infty$ and by (MA4), $A\cap\A$ contains $\mathrm{cl}(z,+\infty)\supset z+\overline{C_f^v}$. Consequently, $A\cap\A \supset z+\delta$ and thus $x+\delta\cap \A=y+\delta$ or $x+\delta\cap \A=y+\mathring{\delta}$, where $\mathring \delta=\R^*_+\nu$ for some $y\in x+\delta$.

 Suppose $x+\delta\cap \A=y+\mathring{\delta}$. Let $z\in y+\mathring{\delta}$. Then by (MA2) applied to $germ_y([y,z]\backslash\{y\})$, $\A\cap A\supset \mathrm{cl}(germ_y([y,z]\backslash\{y\}))\ni y$ because $\mathrm{cl}(germ_y([y,z]\backslash\{y\}))$ contains the closure of $germ_y([y,z]\backslash\{y\})$. This is absurd and thus $\A\cap x+\delta=y+\delta$, with $y\in \A$. Ones sets $y_\nu(x)=y\in\A$ (actually, $y_\nu$ only depends on $\delta$).

One has $\rho_{+\infty}(x+\delta)=\rho_{+\infty}(x)+\delta$ and $y\in \rho_{+\infty}(x)+\delta$. We define $T_\nu(x)$ as the unique element $T$ of $\mathbb{R}_+$ such that $y=\rho_{+\infty}(x)+T\nu$.

Let $\delta^-=-\R_+\nu$ and $x\in \I$. Similarly, one defines $y_\nu^-$ as the first point of $x+\delta^-$ meeting $\A$ and $T^-_\nu(x)$ as the element $T$ of $\R_+$ such that $\rho_{-\infty}(x)=y+T\nu$.

\begin{rque}
Except in the proof of lemma~\ref{lemme distance vectorielle} b), the choice of $\nu$ will not be important.
\end{rque}


\begin{lemme}\label{lemme distance vectorielle}
Let $x\in \mathcal{I}$ and $\nu\in C_f^v$. Let $y=y_\nu(x)$ and $T=T_\nu(x)$.

a)  Then $x\leq y$ and $d^v(x,y)=T\nu$.

b) One has $\rho_{+\infty}(x)\in Y$ if and only if $\rho_{-\infty}(x)\in Y$ if and only if $x\in \mathcal I_0=G.0$. In this case, $\rho_{+\infty}(x)\leq_{Q^\vee} \rho_{-\infty}(x)$.


\end{lemme}

Proof : Let $A$ be an apartment containing $x$ and $+\infty$ and $g\in G$ fixing $+\infty$ such that $A=g^{-1}.\A$. Then $x+\delta$ is the translate of a shortening $\delta'\subset A$ of $\delta$ (which means $\delta'=z+\delta$, with $z\in \delta$). As for all $z'\in z+\delta$, $z\leq z'$, one has $x\leq y$. As $d^v(x,y)=d^v(g.x,g.y)$ and $g_{|A}=\rho_{+\infty}$ one gets a).

For $x\in\mathcal{I}$, there exists $g_-,g_+\in G$ such that $\rho_{-\infty}(x)=g_-.x$ and $\rho_{+\infty}(x)=g_+.x$, which shows the claimed equivalence because $Y=G.0\cap \mathbb{A}$.

Suppose $x\in \mathcal{I}_0$. One chooses $\nu\in Y\cap C_f^v$. Let $S=\lfloor T\rfloor +1$, where $\lfloor.\rfloor$ is the floor function, and $z=\rho_{+\infty}(x)+S\nu\in x+\delta$. Then $d^v(x,z)=d^v(g_+.x,g_+.z)=d^v(\rho_{+\infty}(x),z)=S\nu\in Y^{++}$.

According to paragraph 2.3 of \cite{gaussent2014spherical}, the image $\pi$ of $[x,z]$ by $\rho_{-\infty}$ is a Hecke path of shape $z-\rho_{+\infty}(x)=S\nu$ with respect to $-C^v_f$ (unless the contrary is specified, "Hecke path" will mean with respect to $-C^v_f$). By applying lemma 2.4b) of \cite{gaussent2014spherical} to $\pi$, one gets that $z-\rho_{-\infty}(x)\leq _{Q^\vee} d^v(x,z)=z-\rho_{+\infty}(x)$ and thus $\rho_{+\infty}(x)\leq_{Q^\vee} \rho_{-\infty}(x)$ and one has b). $\square$


\begin{lemme}\label{lemme image réciproque de segments}
Let $\tau:[0,1]\rightarrow \mathcal{I}$ a segment of $\mathcal{I}$ such that $\tau(1)\in \mathbb{A}$ and $\rho_{-\infty}(\tau)$ is a segment of $\mathbb{A}$ satisfying  $(\rho_{-\infty}\circ\tau)'=\nu\in \overline{C^v_f}$. Then $\tau([0,1])\subset \mathbb{A}$ and thus $\rho_{-\infty}\circ \tau=\tau$.
\end{lemme}

Proof : Suppose $\tau([0,1])\not\subset \A$. Let $u=\sup\{t\in [0,1]|\tau(u)\notin \A\}$. Then by the same reasoning as in the proof that "$x+\delta\cap\A=y+\delta$" in the paragraph "Definition of $y_\nu$ and $T_\nu$", $x=\tau(u)\in\A$ One has $\tau(0)\leq x\leq \tau(1)$ (by the same reasoning as in the proof of lemma \ref{lemme distance vectorielle}). 

By remark~\ref{rque axioms MA3 et MA4 modifiés}, there exist an apartment $A=g^{-1}.\A$ with $g\in G$ containing $\mathfrak{R}=-\infty$ and $germ_x([x,\tau(0)])$. By axiom (MA4) (and remark~\ref{rque axioms MA3 et MA4 modifiés}) applied to $\mathfrak{R}=-\infty$ and to $x$, we can suppose that $g$ fixes $\mathrm{cl}(x,-\infty)\supset x-C_f^v$.  
Let $x'\in [x,\tau(0)]\backslash\{x\}$ such that $[x,x']\subset A$. Then $g.x'=\rho_{-\infty}(x')\in x-\R_+\nu\subset x-C_f^v$. Therefore, $x'=g.x'\in \A$, which is absurd. Hence $\tau([0,1])\subset \A$. $\square$





\section{Bounding of T and proof of theorem~\ref{thm inclusion} }\label{sect bounding of T}


One defines $h:\begin{aligned} Q^\vee_{\mathbb{R}}& \rightarrow \mathbb{R}\\
x=\sum x_i\alpha_i^\vee & \mapsto \sum x_i\end{aligned}.$




\begin{lemme}\label{lemme fin des chemins de Hecke longs}


Let $T\in \mathbb{R}_+$, $\mu\in \A$,  $a\in \mathbb{A}$, $\nu\in Y^{++}$ and suppose there exists a Hecke path $\pi$ from $a$ to $a+T\nu-\mu$ of shape $T\nu$. Then 

a) $\mu\in \mathbb{R}_+ Q^\vee_+$. Consequently $h(\mu)$ is well defined.

b) if $T>h(\mu)$, there exists $t$ such that $\pi$ is derivable on $(t,1]$ and $\pi'_{|(t,1]}=\nu$. Furthermore, let $t^*$ be the smallest $t \in [0,1]$ having this property, then $t^*\leq \frac{h(\mu)}{T}$.
\end{lemme}


Proof : The main idea of b) is to use the fact that during the time when $\pi'(t)\neq T\nu$, $\pi'(t)=T\nu-T\lambda(t)$ with $\lambda(t)\in Q^\vee_+\backslash \{0\}$. Hence for $T$ large, $\pi$ decreases quickly for the $Q^\vee$ order, but it cannot decrease too much because $\mu$ is fixed.

Let $t_0=0$, $t_1, \ldots,t_n=1$ a subdivision of $[0,1]$ such that for all $i\in \llbracket 0,n-1\rrbracket $, $\pi_{|(t_i,t_{i+1})}$ is derivable and let $w_i\in W^v$ be such $\pi'_{|(t_i,t_{i+1})}=w_i.T\nu$. If $w_i.\nu=\nu$, one chooses $w_i=1$.

For $i\in \llbracket 0,n-1\rrbracket$, according to lemma 2.4a) of \cite{gaussent2014spherical}, $w_i.\nu =\nu -\lambda_i$, with $\lambda_i \in Q^\vee_+$ and if $w_i\neq 1$, $\lambda_i\neq 0$. One has \[\pi(1)-\pi(0)=T\nu+-\sum_{i=0, w_i\neq 1}^{n-1}(t_{i+1}-t_i)T\lambda_i=T\nu +\mu\]

and one deduces a).


Suppose now $T>h(\mu)$. Let us show that there exists $i\in \llbracket 0,n-1\rrbracket$ such that $w_i=1$. Let $i\in \llbracket 0,n-1\rrbracket$. 

For all $i$ such that $w_i\neq 1$, one has $h(\lambda_i)\geq 1$. Hence $T\sum_{i=0, w_i\neq 1}^{n-1}(t_{i+1}-t_i)\leq h(\mu)$, and for $T>h(\mu)$, $\sum_{i=0, w_i\neq 1}^{n-1}(t_{i+1}-t_i)<1=\sum_{i=0}^{n-1}(t_{i+1}-t_i)$ and thus there exists $i\in \llbracket 0,n-1\rrbracket$ such that $w_i=1$.

By remark~\ref{rque chemins de Hecke}, if $w_i=1$ for some $i$, then $w_j=1$ for all $j\geq i$. This shows the existence of $t^*$. We also have $t^*\leq \sum_{i=0, w_i\neq 1}^{n-1}(t_{i+1}-t_i)$ and hence the claimed inequality follows. $\square$


\begin{rque}\label{rque condition sur mu}

 Part a) of lemma~\ref{lemme fin des chemins de Hecke longs} shows that for $x\in \mathcal{I}$, $\rho_{-\infty}(x)-\rho_{+\infty}(x)\in \mathbb{R}_+Q^\vee_+$.
\end{rque}




From now on and until the end of this subsection, $\nu$ will be a fixed element of $C_f^v$.

\begin{cor}\label{cor points ayant meme image par les rétractions}
 Let $x\in\mathcal{I}$ such that $\rho_{+\infty}(x)=\rho_{-\infty}(x)$. Then $x\in \mathbb{A}$. Therefore, $\forall z\in \mathbb{A}$, $\rho_{+\infty}^{-1}(\{z\})\cap\rho_{-\infty}^{-1}(\{z\})=\{z\}$.
\end{cor}


Proof :  Let $x\in\mathcal{I}$ such that $\rho_{+\infty}(x)=\rho_{-\infty}(x)$. Let $\nu\in C_f^v$. Let $y=y_\nu(x)$ and $T=T_\nu(x)$ be as in the paragraph "Definition of $y_\nu$ and $T_\nu$". Suppose $x\notin \mathbb{A}$. Then $T>0$. Let $g\in G$ fixing $+\infty$ and such that $x\in g^{-1}.\A$. Let $\tau:[0,1]\rightarrow g^{-1}.\A$ defined by $\tau(t)=g^{-1}.(\rho_{+\infty}(x)+tT\nu)$ for all $t\in [0,1]$. Then $\tau$ is a segment going from $x$ to $y$ and $d^v(x,y)=T\nu$. Therefore, the image $\pi$ of $\tau$ by $\rho_{-\infty}$ of $[x,y]$ is a Hecke path from $\rho_{-\infty}(x)$ to $y=\rho_{+\infty}(x)+T\nu=\rho_{-\infty}(x)+T\nu$, of shape $T\nu$. By lemma \ref{lemme fin des chemins de Hecke longs}, $\pi$ is a segment of $\mathcal{I}$ such that $\pi'=T\nu$. As $\tau(1)=y\in \A$, one can apply lemma \ref{lemme image réciproque de segments} and one gets that $\pi([0,1])\in \mathbb{A}$. This is absurd because we supposed that $x\notin \mathbb{ A}$ and hence $x\in \mathbb{A}$. $\square$

\begin{cor}\label{corollaire majoration de T}
Let $\mu\in Q^\vee_\R$. Then for all $x\in \I$ such that $\rho_{-\infty}(x)-\rho_{+\infty}(x)=\mu$, $T_\nu(x)\leq h(\mu)$, where $T_\nu(x)$ is as defined in the paragraph "Definition of $y_\nu$ and $T_\nu$".
\end{cor}

Proof : Let $y=y_\nu(x)$. Let $\pi$ be the image by $\rho_{-\infty}$ of $[x,y]$. This is a Hecke path from $\rho_{-\infty}(x)$ to $y=\rho_{+\infty}(x)+T\nu$, of shape $T\nu$, with $T=T_\nu(x)$. The minimality of $T$ and lemma \ref{lemme image réciproque de segments} imply that $\pi'(t)\neq \nu$ for all $t\in [0,1]$, where $\pi'$ is derivable. By applying lemma \ref{lemme fin des chemins de Hecke longs}, we deduce that $T\leq h(\mu)$. $\square$




\begin{lemme}\label{lemme_rétraction et distance vectorielle}
Let $x\in \mathcal{I}$ such that $y_\nu^-(x)\in C^v_f$. Then $0\leq x$ and $\rho_{-\infty}(x)=d^v(0,x)$.
\end{lemme}

Proof : Let $y^-=y^-_\nu(x)$. Let $A$ be an apartment containing $x$ and $-\infty$. By (MA4) there exists $g\in G$ such that $A=g^{-1}.\A$ and $g$ fixes $\mathrm{cl}(y,-\infty)\supset y-\overline{C_f^v}\ni 0$. Then $g.x-g.y^-=\rho_{-\infty}(x)-y^-=T^-\nu\in C_f^v$ and $g.y-g.0=y$. Thus $g.x-g.0=\rho_{-\infty}(x)\in C_f^v$ and we can conclude by remark~\ref{rque caractérisation distance vectorielle}. $\square$




\begin{thm}\label{thm inclusion}
Let $\mu\in \A$. Then if $\mu\notin Q_{\R}^\vee$, $\rho_{+\infty}^{-1}(\{\lambda+\mu\})\cap \rho_{-\infty}^{-1}(\{\lambda\})$ is empty for all $\lambda\in \A$. If $\mu\in Q^\vee_{\R}$, then for $\lambda\in \A$ sufficiently dominant, $\rho_{+\infty}^{-1}(\{\lambda+\mu\})\cap \rho_{-\infty}^{-1}(\{\lambda\})\subset B^v(0,\lambda)\cap \rho_{+\infty}^{-1}(\{\lambda+\mu\}) $. 
\end{thm}

Proof : The condition on $\mu$ comes from remark~\ref{rque condition sur mu}. 

Suppose $\mu\in Q^\vee_\R$.
Let $\lambda\in \A$. Let $y^-=y^-_\nu$ and $T^-=T_\nu^-$. By corollary~\ref{corollaire majoration de T}, if $x\in \rho_{+\infty}^{-1}(\{\lambda+\mu\})\cap \rho_{-\infty}^{-1}(\{\lambda\})$, then $y^-(x)\in [\lambda-T^-(x)\nu, \lambda]\subset \lambda-[0,h(\mu)]\nu$. 
For all $i\in I$, $\alpha_i([0,h(\mu)]\nu)$ is bounded, Consequently $\lambda$ sufficiently dominant, $\alpha_i(\lambda-[0,h(\mu)]\nu)\subset \mathbb{R}^*_+$ for all $i\in I$.  For such a $\lambda$,  $y^-\big(\rho_{+\infty}^{-1}(\{\lambda+\mu\})\cap \rho_{-\infty}^{-1}(\{\lambda\})\big)\subset C^v_f$. We conclude the proof with lemma \ref{lemme_rétraction et distance vectorielle}.
 $\square$

\section{Study of "translations" of $\I$ and proof of theorem~\ref{thm invariance des cardinaux}}\label{sect translations}

Let $\A_{in}=\bigcap_{i\in I} \ker \alpha_i$.

In this subsection we introduce some kinds of "translation" of $\I$ of an inessential vector. It will be very useful to "generalize theorem from $Y$ to $Y+\A_{in}$" by getting rid of the inessential part. First example of this technique will be corollary~\ref{cor finitude des boules} which generalizes a theorem of \cite{gaussent2014spherical}. Then we study elements of $G$ inducing a translation on $\A$. We show that they commute with $\rho_{+\infty}$ and $\rho_{-\infty}$. We then can see that for fixed $\mu\in\Q^\vee$, the $\rho_{+\infty}^{-1}(\{\lambda+\mu\})\cap\rho_{-\infty}^{-1}(\{\lambda\})$ for $\lambda\in Y+\A_{in}$ are some translated of  $\big(\rho_{+\infty}^{-1}(\{\mu\})\cap\rho_{-\infty}^{-1}(\{0\})\big)$  which enables to show theorem~\ref{thm invariance des cardinaux} by using theorem~\ref{thm inclusion} and corollary~\ref{cor finitude des boules}.

\begin{lemme}\label{lemme partie inessentielle}
 Let $\nu\in\A_{in}$ and $a\in \I$. Then $|B^v(a,\nu)|=1$. Moreover, if $a\in \A$, $B^v(a,\nu)=\{a+\nu\}$. 

\end{lemme} 

Proof : Suppose first $a\in\A$. Let $x\in B^v(a,\nu)$. Let $g\in G$ such that $x,a\in g^{-1}.\A$ and $g.x-g.a=\nu$. Let $\tau:[0,1]\rightarrow \I$ defined by $\tau(t)=g^{-1}.(g.a+(1-t)\nu)$. Then $\pi=\rho_{-\infty}\circ \tau$ is a Hecke path of shape $-\nu$ and in particular,  it is a $-\nu$-path. For all $t$ where $\pi$ is derivable, there exists $w(t)\in W^v$ such that $\pi'(t)=-w(t).\nu=-\nu$. As $W^v$ acts trivially on $\A_{in}$, $\pi'(t)=-\nu$ for all such $t$ and thus $\pi$ is derivable on $[0,1]$ and $\pi'=-\nu$. As $\tau(1)=a\in \A$, one can apply lemma~\ref{lemme image réciproque de segments} and we get that $\tau([0,1])\subset \A$. Therefore, $x\in \A$ and there exists $w\in W^v$ such that $x-a=w.\nu=\nu$.

\vspace{3mm}

This lemma enables us to define some kinds of translation of a inessential vector. Let $\nu\in \A_{in}$. Let $\tau_\nu:\I\rightarrow \I$ which associates to $x\in \I$ the unique element of $B^v(x,\nu)$. Then we have the following lemma : 

\begin{lemme}\label{lemme propriétés des translations}
\begin{enumerate}
Let $\nu\in \A_{in}$ and $\tau=\tau_\nu$. Then :

\item For all $x\in \A$, $\tau(x)=x+\nu$.\label{item tau restreint à A}

\item For all $g\in G$, $g\circ\tau=\tau\circ g$. In particular, for all $x\in \I$, if $x$ is in an apartment $A$,
 then $\tau(x)\in A$, and if $x=g.a$ with $g\in G$ and $a\in\A$, then $\tau(x)=g.(x+\nu)$.\label{item commutation de tau}

\item The application $\tau$ is a bijection, its inverse being $\tau_{-\nu}$.\label{item bijectivité des translations}


\item The application $\tau$ commutes with $\rho_{+\infty}$ and $\rho_{-\infty}$\label{item commutation translation retraction}

\item Let $x\in \I$ and $\lambda\in \overline{C_f^v}$, then $\tau(B^v(x,\lambda))=B^v(\tau(x),\lambda)=B^v(x,\lambda+\nu).$\label{item translation d'une boule}

\end{enumerate}
\end{lemme}

Proof : part~\ref{item tau restreint à A} is a part of lemma~\ref{lemme partie inessentielle}.


Choose $x\in \I$ and $g\in G$. Then $d^v(x,\tau(x))=\nu$ and thus $d^v(g.x,g.\tau(x))=\nu$. Consequently $g.\tau(x)=\tau(g.x)$ by definition of $\tau$. Let $A$ be an apartment containing $x$, $A=h.\A$, with $h\in G$. Then $\tau(x)=\tau(h.a)$ with $a\in \A$, hence $\tau(x)=h.(x+\nu)\in A=h.\A$ and we have~\ref{item commutation de tau}.

Let $x\in\I$, $x=g.a$ with $a\in \A$. By part~\ref{item commutation de tau} applied to $\tau$ and $\tau_{-\nu}$, one has $\tau_{-\nu}(\tau(g.a))=g.\tau_{-\nu}(\tau(a))=g.a$ and thus $\tau_{-\nu}\circ\tau=\Id$. By symmetry, $\tau\circ\tau_{-\nu}=\Id$, which shows~\ref{item bijectivité des translations}.

Let $x\in \I$ and $g\in G$ fixing $+\infty$ such that $g.x=\rho_{+\infty}(x)$. 
Then $\tau(x)\in g^{-1}.\A$, thus $g.\tau(x)=\rho_{+\infty}(\tau(x))$ and by
 part~\ref{item commutation de tau}, $g.\tau(x)=\tau(g.x)$. Hence, $\tau$ and
  $\rho_{+\infty}$ commute and by the same reasoning, this is also true for $\tau$ and $\rho_{-\infty}$.

Let $x\in\I$ and $\lambda\in \overline{C_f^v}$. Let $u\in B^v(x,\lambda)$. There exists $g\in G$ such that $x,u\in g^{-1}.\A$ and $g.u-g.x=\lambda$. Then $g.\tau(u)-g.x=\tau(g.u)-g.x=\lambda+\nu$. Therefore, $\tau(B^v(x,\lambda))\subset B^v(x,\lambda+\nu)$. Applying this result with $\tau_{-\nu}$ yields $\tau_{-\nu}(B^v(x,\lambda+\nu))\subset B^v(x,\lambda)$ and thus $\tau(B^v(x,\lambda))=B^v(x,\lambda+\nu)$.

One has $g.\tau(u)-g.\tau(x)=(g.u+\nu)-(g.x+\nu)=\lambda$ and thus $\tau(B^v(x,\lambda))\subset B^v(\tau(x),\lambda)$. Again, by considering $\tau_{-\nu}$, we have that $\tau(B^v(x,\lambda))=B^v(\tau(x),\lambda)$. $\square$


 
 
\begin{cor}\label{cor finitude des boules}
Let $\lambda\in \A$ and $\mu\in Y+\A_{in}$. Then for all $\lambda_{in}\in \A_{in}$, $\tau_{\lambda_{in}}\big(B^v(0,\lambda)\cap \rho_{+\infty}^{-1}(\{\mu\}\big)=B^v(0,\lambda+\lambda_{in})\cap \rho_{+\infty}^{-1}(\{\mu+\lambda_{in}\})$. In particular, for all $\lambda\in Y+\A_{in}$, $B^v(0,\lambda)\cap \rho_{+\infty}^{-1}(\{\mu+\lambda\})$ is finite and is empty if $\mu\notin Q^\vee_-$.

\end{cor}

Proof : The first assertion is a consequence of lemma~\ref{lemme propriétés des translations} part~\ref{item bijectivité des translations}, \ref{item commutation translation retraction} and \ref{item translation d'une boule}.

Let $\lambda\in Y+\A_{in}$ and $\lambda_{in}\in \A_{in}$ such that $\tau(\lambda)\in Y$, 
with $\tau=\tau_{\lambda_{in}}$. Then 
$\tau\big(B^v(0,\lambda)\cap\rho_{+\infty}^{-1}(\{\lambda+\mu\})\big)=B^v(0,\tau(\lambda))\cap\rho_{+\infty}^{-1}$ $(\{\tau(\lambda+\mu )\})$. Consequently, one can suppose $\lambda\in Y$. 

Suppose $B^v(0,\lambda)\cap \rho_{+\infty}^{-1}(\{\mu+\lambda\})$ is nonempty and let $x$ be in this set. Then there exists $g,h\in G$ such that $g.x=\lambda$ and $h.x=\mu+\lambda$. Thus $\lambda+\mu=h.g^{-1}.\lambda\in \I_0\cap \A=Y$ and therefore, $\mu\in Y$. We can now conclude because the finiteness and the condition $\mu$ are shown in \cite{gaussent2014spherical}, part 5 : if $\lambda, \mu\in Y$ then $B^v(0,\lambda)\cap \rho_{+\infty}^{-1}(\{\mu\})$ is finite and it is empty if $\mu\notin Q^\vee_-$ (the cardinals of these sets correspond to the $n_\lambda(\nu)$ of this part).
$\square$

\vspace{3mm}

We now show a lemma similar to lemma~\ref{lemme propriétés des translations} part~\ref{item commutation translation retraction} for translations of $G$ : 

\begin{lemme}\label{lemme commutation des translations et rétractions}
Let $n\in G$ inducing a translation on $\A$. Then $n\circ\rho_{+\infty}=\rho_{+\infty}\circ n$ and $n\circ \rho_{-\infty}=\rho_{-\infty}\circ n$.
\end{lemme}

Proof : 
Let $x\in\mathcal{I}$ and $A$ be an apartment containing $x$ and $+\infty$. Then $n.A$ is an apartment containing $+\infty$. Let $\phi:A\rightarrow \mathbb{A}$ an isomorphism fixing $+\infty$. We have $n.x\in n.\mathbb{A}$, and $n\circ\phi\circ n^{-1}:n.A\rightarrow \mathbb{A}$ fixes $+\infty$. Hence $\rho_{+\infty}(n.x)=n\circ\phi\circ n^{-1}(n.x)=n\circ\phi(x)=n\circ\rho_{+\infty}(x)$ and thus $n\circ \rho_{+\infty}=\rho_{+\infty}\circ n$. By the same reasoning applied to $\rho_{-\infty}$, we get the lemma. $\square$
 
 
 

\begin{lemme}\label{lemme y- des translatés}
Let $n\in G$ inducing a translation on $\mathbb{A}$. Let $\lambda_{in}\in \A_{in}$. Set $\tau=\tau_{\lambda_{in}}\circ n$. Let $\nu\in C_f^v$ and $y^-=y_\nu^-$. Then $\tau\circ y^-=y^-\circ \tau$.
\end{lemme}

Proof : If $x\in \mathbb{A}$, then $y_-(x)=x$, $y_-(\tau(x))=\tau(x)$ and there is nothing to prove. 

Suppose $x\notin \mathbb{A}$. 
Then $[x,y^-(x)]\backslash \{y^-(x)\}\subset (x-\R_+\nu)\backslash \mathbb{A}$,
 thus $\tau([x,y^-(x)]\backslash\{y_-(x)\})\subset (\tau(x)-\R_+\nu)\backslash \mathbb{A}$ and $\tau(y^-(x))\in \mathbb{A}$. $\square$
 




\begin{thm}\label{thm invariance des cardinaux}
Let $\mu\in \A$ and $\lambda\in Y+\A_{in}$. One writes $\lambda=\lambda_{in}+\Lambda$, with $\lambda_{in}\in\A_{in}$ and $\Lambda\in Y$. Let $n\in G$ inducing the translation of vector $\Lambda$ and $\tau=\tau_{\lambda_{in}}\circ n$. Then $\rho_{+\infty}^{-1}(\{\lambda+\mu\})\cap\rho_{-\infty}^{-1}(\{\lambda\})=n\big(\rho_{+\infty}^{-1}(\{\mu\})\cap\rho_{-\infty}^{-1}(\{0\})\big)$. Therefore theses sets are finite and if $\mu\notin Q^\vee_-$ these sets are empty.

\end{thm}

Proof : First assertion is a consequence of lemma~\ref{lemme commutation des translations et rétractions} and lemma~\ref{lemme propriétés des translations} part~\ref{item commutation translation retraction}. Then lemma~\ref{lemme distance vectorielle} b) shows that these sets are empty unless $\mu\in Q^\vee_+$. 

By theorem~\ref{thm inclusion}, for $\lambda'\in Y$ sufficiently dominant, $\rho_{+\infty}^{-1}(\{\lambda'+\mu\})\cap\rho_{-\infty}^{-1}(\{\lambda'\})\subset B^v(0,\lambda')\cap \rho_{+\infty}^{-1}(\{\lambda'+\mu\})$, which is a finite set by \cite{gaussent2014spherical} part 5 (or by corollary~\ref{cor finitude des boules}). $\square$








\section{Description of $Y^{++}$}\label{sect Y^++}

In this section we show that $Y^{++}$ is a finitely generated monoid, which will be useful to prove theorem~\ref{thm égalité des ensembles bis}.



Let $l\in \mathbb{Z}_{>0}$. Let us define a binary relation $\prec$ on $\mathbb{Z}_{>0}$. Let $x,y\in \mathbb{Z}_{>0}^l$,, $x=(x_1,\ldots,x_l),\ y=(y_1,\ldots,y_l)$ then one says $x\prec y$ if $x\neq y$  and for all $i\in \llbracket 1,l\rrbracket$, $x_i\leq y_i$.
 
\begin{lemme}\label{lemme ensembles d'incomparables}
Let $l\in \mathbb{Z}_{>0}$ and $F$ a subset of $\mathbb{Z}_{>0}$ satisfying property (INC(l)) : 

for all $x,y\in F$, ($x\neq y$) implies ($x$ and $y$ are not comparable for $\prec$). 

 Then $F$ is finite.
\end{lemme}

Proof : this  is clear for $l=1$ because a set $F$ satisfying INC(1) is a singleton or $\emptyset$. 

Suppose that $l>1$ and that we have proven that all set satisfying INC(l-1) is finite.

 Let $F$ be a set  satisfying INC(l) and suppose $F$ infinite. Let $(\lambda_n)_{n\in \mathbb{Z}_{\geq 0}}$ be an injective sequence of $F$. One write $(\lambda_n)=(\lambda_n^1,\ldots,\lambda_n^l)$. Let $(m_n)=(\min(\lambda_n))_{n\in \N}$ and $M=\max \lambda_{0}$. Then for all $n\in \mathbb{Z}_{\geq 0}$, $m_n\leq M$ (if $m_n>M$, we would have $\lambda_{0}\prec \lambda_n$). Maybe extracting a sequence of $\lambda$, one can suppose that $(m_n)=(\lambda_n^i)$ for some $i\in \llbracket 1, l\rrbracket$ and that $(m_n)$ is constant and equal to $m_0$. For $x\in \mathbb{Z}_{\geq 0}^l$, we define $\tilde{x}=(x_j)_{j\in \llbracket 1,l\rrbracket\backslash \{i\}}\in \mathbb{Z}_{\geq 0}^{l-1}$. 
 
 Set $\tilde{F}=\{\tilde{\lambda_n}|n\in\mathbb{Z}_{\geq 0}\}$. The set $\tilde{F}$ satisfies INC(l-1). By the induction hypothesis, $\tilde{F}$ is finite and thus $F$ is finite, which is absurd. Hence $F$ is finite and the lemma is proven. $\square$

\begin{lemme}\label{lemme description de Y^{++}}
There exists a finite set $E$ such that $Y^{++}=\sum_{e\in E}\mathbb{Z}_{\geq 0}e$.
\end{lemme}

Proof : the set $Y_{in}=Y\cap\A_{in}$ is a lattice in the vectorial space it spans. Consequently, it is a finitely generated $\Z$-module and thus a finitely generated monoid. Let $E_1$ a finite set generated $Y_{in}$ as a monoid.

 Let $Y_{\succ 0}=Y^{++}\backslash Y_{in}$. Let $\mathcal{P}=\{a\in Y_{\succ 0}|a\neq b+c\ \forall b,c\in Y_{\succ 0}\}$. Let $\alpha:Y^{++}\rightarrow \mathbb{Z}_{\geq 0}^{I}$ such that 
 $\alpha(x)=(\alpha_i(x))_{i\in I}$ for all $x\in Y^{++}$. Let $a,b\in \mathcal{P}$.
  If $\alpha(a)\prec \alpha(b)$, then $b=b-a+a$, with $a,b-a\in \mathcal{P}$, which is absurd and by symmetry we deduce that $\alpha(a)$ and $\alpha(b)$ are not comparable for $\prec$. 
Therefore, by lemme~\ref{lemme ensembles d'incomparables}, $\alpha(\mathcal{P})$ is finite. Let $E_2$ be a finite set of $Y_{\succ 0}$ such that $\alpha(\mathcal{P})=\{\alpha(x)|x\in E\}$. Then $Y^{++}=\sum_{e\in E_2}\mathbb{Z}_{\geq 0}e+Y_{in}=\sum{e\in E}\N e$, where $E=E_1\cup E_2$. $\square$

 

\section{Proof of theorem~\ref{thm égalité des ensembles bis}}\label{sect proof of final theorem}

The basic idea of the proof of theorem~\ref{thm égalité des ensembles bis}, is that if $\mu\in Q^\vee$ there exist a finite set $F\subset Y^{++}$ such that for all $\lambda\in Y^{++}+\A_{in}$, $\rho_{+\infty}^{-1}(\{\lambda+\mu\})\cap B^v(0,\lambda) \subset \bigcup_{f\in F|\lambda-f\in\overline{C^v_f}} \rho_{+\infty}^{-1}(\{\lambda+\mu\})\cap B^v(\lambda-f,f)$ (this is lemma~\ref{lemme majoration du cardinal des boules}, which generalizes lemma~\ref{lemme distance finie à l'appartement} and uses section~\ref{sect Y^++}). Then we use section~\ref{sect translations} to show  that $\rho_{+\infty}^{-1}(\{\lambda+\mu\})\cap B^v(\lambda-f,f)$ is the image of $\rho_{+\infty}^{-1}(\mu+f)\cap B^v(0,f)$ by a "translation" $\tau_{\lambda-f}$ of $G$ of vector $\lambda-f$ (which means that $\tau_{\lambda-f}$ induces the translation of vector $\lambda-f$ on $\mathbb{A}$). We fix $\nu\in C_f^v$ and set $y^-=y_\nu^-$. By lemma~\ref{lemme y- des translatés},  \[y^-\big(\rho_{+\infty}^{-1}(\{\lambda+\mu\})\cap B^v(0,\lambda)\big)\subset\bigcup_{f\in F}\tau_{\lambda-f}\circ y^-\big(B^v(0,f)\cap \rho_{+\infty}^{-1}(\{\mu +f\})\big).\] According to part 5 of \cite{gaussent2014spherical}, for all $f, \mu \in Y$, $B^v(0,f)\cap \rho_{+\infty}^{-1}(\{\mu +f\})$ is finite. Consequently, for $\lambda$ sufficiently dominant, $\bigcup_{f\in F}\tau_{\lambda-f}\big(y_-(B^v(0,f)\cap \rho_{+\infty}^{-1}(\{\mu +f\}\big)\subset C^v_f$  and one conclude by using lemma~\ref{lemme_rétraction et distance vectorielle}. 
 



\begin{lemme}\label{lemme distance finie à l'appartement}
Let $\mu\in Q^\vee_-$ and $H=-h(\mu)+1\in \mathbb{Z}_{\geq 0}$. Let $x\in \mathcal{I}$ and $a\in Y$ such that $d^v(a,x)=T\nu$, with $T\geq H$ and $\rho_{+\infty}(x)=a+T\nu +\mu$. Let $g\in G$ such that $g.a=a$ and $g.x=T\lambda+a$. Then $g$ fixes $[a,a+(T-H)\nu]$ and in particular $x\in B^v(a+(T-H)\nu,H\nu)$.
\end{lemme}

Proof : Let $\tau:[0,1]\rightarrow \mathbb{A}$ defined by $\tau(t)=a+(1-t)T\nu$. The main idea is to apply lemma \ref{lemme fin des chemins de Hecke longs} to $\rho_{+\infty}(g.\tau)$ but we cannot do it directly because $\rho_{+\infty}(\tau)$ is not a Hecke path with respect to $-C^v_f$. Let $\mathbb{A}'$ be the vectorial space $\mathbb{A}$ equipped with a structure of apartment of $-\mathbb{A}$: the fundamental chamber of $\mathbb{A}'$ is $C^{v,}_f=-C^v_f$ etc ... Let $\mathcal{I}'$ be the set $\mathcal{I}$, whose apartments are the $-A$ where $A$ runs over the apartment of $\mathbb{A}$. Then $\mathcal{I}'$ is a hovel of standard apartment $\mathbb{A}'$. We have $0\leq x$ in $\mathcal{I}$ and so $x\leq'0$ in $\mathcal{I}'$. 
 
Then the image $\pi$ of $g.\tau$ by $\rho_{+\infty}$ is a Hecke path of shape $-T\nu$ from $\rho_{+\infty}(x)=a+T\nu +\mu$ to $a$. By lemma \ref{lemme fin des chemins de Hecke longs}, for $t>-h(\mu)/T$, $\pi'(t)=-T\nu$, and thus $\rho_{+\infty}(g.\tau(t))=\tau(t)$. According to lemma \ref{lemme image réciproque de segments}, if $t>-h(\mu)/T$, $\tau(t)\in \mathbb{A}$, thus $\rho_{+\infty}(g.\tau(t))=g.\tau(t)=\tau(t)$ and consequently $g$ fixes  $\tau(t)$. In particular, $g$ fixes $[a,a+(T-H)\nu=\tau(\frac{H}{T})]$, and $d^v(a+(T-H)\nu,x)=d^v(g^{-1}.(a+(T-H)\nu),g^{-1}.x)=d^v(a+(T-H)\nu,a+T\nu)=H\nu$. $\square$











\vspace{5mm}

Let $E$ be as in the lemma~\ref{lemme description de Y^{++}}. 
 
\begin{lemme}\label{lemme majoration du cardinal des boules}
 Let $\mu \in Q^\vee_-$. Let $H=-h(\mu)+1$. Let $\lambda\in Y^{++}$. We fix a writing  $\lambda=\lambda_{in}+\sum_{e\in E}\lambda_e e$, with $\lambda_{in}\in Y_{in}$ and $\lambda_e\in \mathbb{Z}_{\geq 0}$ for all $e\in E$. Let $J=\{e\in E | \lambda_e \geq H\}$. Then $B^v(0,\lambda)\cap \rho_{+\infty}^{-1}(\{\lambda+\mu\})\subset B^v(\lambda-H\sum_{e\in J} e-\sum_{e\notin J}\lambda_e e, H\sum_{e\in J}e+\sum_{e\notin J}\lambda_ e)$. 
\end{lemme}
 
Proof : Let $m=|J|$. One writes $J=\{e_1,\ldots,e_m\}$ and for $i\in \llbracket 1,m\rrbracket$, one writes $\lambda_i$ instead or $\lambda_{e_i}$.
 
Let $g\in G$ fixing $0$ and such that $g.x=\lambda$. For $i\in \llbracket 1,m\rrbracket$, let $x_i=\sum_{j=1}^i (\lambda_i-H)e_i$ and let $x_{m+1}=x$.
 
Let $i\in \llbracket 0,m-1\rrbracket$ and suppose we have proven that $g$ fixes $\bigcup_{j=0}^{i-1}[x_j,x_{j+1}]$, with $x_0=0$.
 
 
For $j\in\llbracket i,m\rrbracket$, let $z_j=g^{-1}.(x_{i}+\sum_{k=i+1}^j \lambda_k e_k)$, and $z_{m+1}=x$ (one has $z_i=x_i$ by the induction hypothesis or by the fact that $g$ fixes $0$ if $i=0$).

Let us show that $\rho_{+\infty}(z_{i+1})=\lambda_{i+1}e_{i+1}+\mu'$, with $\mu\leq \mu' \leq 0$.

According to lemma 2.4.b) of \cite{gaussent2014spherical} (adapted because one considers Hecke paths with respect to $C^v_f$), one has : 

$\rho_{+\infty}(z_{i+1})-\rho_{+\infty}(z_i)=\rho_{+\infty}(z_{i+1})-x_i\leq_{Q^\vee}d^v(z_i,z_{i+1})=\lambda_{i+1}e_{i+1}$, 
 
for all $j\in \llbracket i+1, m-1\rrbracket$, $\rho_{+\infty}(z_{j+1})-\rho_{+\infty}(z_j)\leq_{Q^\vee} d^v(z_j,z_{j+1})= \lambda_{j+1}e_{j+1}$ and
 
$\rho_{+\infty}(x_{m+1})-\rho_{+\infty}(x_m)\leq_{Q^\vee} H(e_1+\ldots + e_i)+\sum_{e\notin J}\lambda_e e$.


Therefore one has :   \[\begin{aligned} \lambda+\mu =\rho_{+\infty}(z_{m+1})& \leq_{Q^\vee}\rho_{+\infty}(z_m)+H(e_1+\ldots+ e_i)+\sum_{e\notin J}\lambda_e e\\&\leq_{Q^\vee} \ldots \\ &\leq_{Q^\vee} \rho_{+\infty}(z_{i+1})+\lambda_m e_m+\ldots+\lambda_{i+2}e_{i+2}+H(e_1+\ldots+e_{i})+f\\ &\leq_{Q^\vee}z_i+\lambda_m e_m+\ldots+\lambda_{i+1}e_{i+1}+H(e_1+\ldots+e_i)+\sum_{e\notin J}\lambda_e e=\lambda.\end{aligned}\]

Thus $\lambda_{i+1}e_{i+1}+\mu\leq_{Q^\vee}\rho_{+\infty}(z_{i+1})\leq_{Q^\vee}\lambda_{i+1}e_{i+1}$, so $\rho_{+\infty}(z_{i+1})=\lambda_{i+1}e_{i+1}+\mu'$, with $\mu\leq_{Q^\vee} \mu'\leq_{Q^\vee} 0$.


One has $H=\max\{-h(u)+1|\mu\leq_{Q^\vee}u\leq_{Q^\vee}0\}$. Hence according to lemma~\ref{lemme distance finie à l'appartement}, $g$ fixes $[x_i,z_{i+1}-He_{i+1}=x_{i+1}]$. 
 
By induction, $g$ fixes $\bigcup_{j=0}^{m-1}[x_j,x_{j+1}]$ and in particular, $g$ fixes $x_m=\lambda-H(e_1+\ldots+e_m)-\sum_{e\notin J}\lambda_e e$ which is the desired result. $\square$
 


    
 \begin{thm}\label{thm égalité des ensembles bis}
 Let $\mu\in Q^\vee$. Then for $\lambda\in Y^{++}+\A_{in}$ sufficiently dominant, $B^v(0,\lambda)\cap \rho_{+\infty}^{-1}(\{\lambda+\mu\})=\rho_{-\infty}^{-1}(\{\lambda\})\cap\rho_{+\infty}^{-1}(\{\lambda+\mu\})$.
 \end{thm}
 
 
Proof : Theorem \ref{thm inclusion} yields one inclusion. It remains to show that $\rho_{+\infty}^{-1}(\{\lambda+\mu\})\cap B^v(0,\lambda)\subset\rho_{+\infty}^{-1}(\{\lambda+\mu\})\cap \rho_{-\infty}^{-1}(\{\lambda\})$ 
for $\lambda$ sufficiently dominant.

\vspace{3mm}
Let $H=-h(\mu)+1$ and $F=\{\sum_{e\in E}\nu_e e|(\nu_e)\in \llbracket 0,H-1\rrbracket^E\}$. This set is finite. Let $\lambda\in Y^{++}+\A_{in}$, $\lambda=\lambda_{in}+\Lambda$, with $\lambda_{in}\in \A_{in}$ and $\Lambda\in Y^{++}$.

Let $x\in B^v(0,\lambda)\cap \rho_{+\infty}^{-1}(\{\lambda+\mu\})$. Then by lemma~\ref{lemme majoration du cardinal des boules}, there exists $f\in F$ such that $\lambda-f\in \overline{C^v_f}$ and $x\in B^v(\lambda-f,f)$. Let $n$ an element of $G$ inducing the translation of vector $\Lambda-f=\lambda-\lambda_{in}+f$ on $\A$ and $\tau_{\lambda,f}=\tau_{\lambda_{in}}\circ n$. Then $x\in\tau_{\lambda,f}(B_f)$ where $B_f=B^v(0,f)\cap \rho_{+\infty}^{-1}(\{\mu +f\})$.

Let $B=\bigcup_{f\in F}B_f$. Then one has proven that \[B^v(0,\lambda)\cap \rho_{+\infty}^{-1}(\{\lambda+\mu\})\subset \bigcup_{f\in F}\tau_{\lambda,f}(B).\]

By part 5 of \cite{gaussent2014spherical} (or corollary~\ref{cor finitude des boules}), $B_f$ is finite for all $f\in F$ and thus $B=\bigcup_{f\in F}B_f$ is finite.  Let $\nu\in C_f^v$ and $y^-=y_\nu^-$. Then $y^-(B)$ is finite and for $\lambda$ sufficiently dominant, $\bigcup_{f\in F}\tau_{\lambda,f}\circ y^-(B) \subset{C^v_f}$. Moreover, according to lemma~\ref{lemme y- des translatés}, $\bigcup_{f\in F}\tau_{\lambda,f}\circ y^-(B)=\bigcup_{f\in F}y^-\circ \tau_{\lambda,f}(B)$. Hence \[y^-\big(B^v(0,\lambda)\cap \rho_{+\infty}^{-1}(\{\lambda+\mu\})\big)\subset C^v_f\] for $\lambda$ sufficiently dominant. Eventually one concludes with lemma~\ref{lemme_rétraction et distance vectorielle}. $\square$




 
\bibliography{bibliographie}
\bibliographystyle{plain}
 \end{document} 



